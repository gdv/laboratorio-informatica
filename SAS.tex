\documentclass[12pt]{beamer}
\usepackage{mathptmx}
\usepackage{helvet}
%\usepackage{courier}
\usepackage{fancyvrb}
\RecustomVerbatimCommand{\VerbatimInput}{VerbatimInput}{frame=single, 
%fontsize=\scriptsize, 
%fontfamily=helvetica,
numbersep=1mm, 
numbers=left, 
formatcom=\color{orange}}
%\usepackage[bitstream-charter]{mathdesign}
\usepackage{listings}
\lstset{commentstyle=\color{orange}, keywordstyle=\color{yellow} }

\usetheme[⟨options⟩]{Boadilla}
\useoutertheme{split}
\usecolortheme{albatross}
%\setbeamercolor*{normal text}{fg=white!65!black,bg=blue!80!black}
%\usefonttheme{professionalfont}
\usefonttheme[onlymath]{serif}

%\usepackage{wallpaper}
%\CenterWallPaper{0.8}{logounimib}


\setbeamertemplate{blocks}[rounded][shadow=false]
\setbeamertemplate{navigation symbols}{}

\setbeamercolor*{structure}{fg=green!75!black,bg=blue!70!white}
\setbeamercolor*{normal text}{fg=green!65!black,bg=blue!80!black}
\setbeamercolor{palette primary}{use={structure,normal text},fg=green,bg=structure.bg!75!black}
\setbeamercolor{palette secondary}{use={structure,normal text},fg=structure.fg,bg=structure.bg!60!black}
\setbeamercolor{palette tertiary}{use={structure,normal text},fg=structure.fg,bg=structure.bg!45!black}
\setbeamercolor{palette quaternary}{use={structure,normal text},fg=green,bg=structure.bg!75!black}
\setbeamercolor*{example text}{fg=green!65!black}
\setbeamercolor*{block body}{bg=structure.bg!90!black}
\setbeamercolor*{block body alerted}{bg=structure.bg!90!black}
\setbeamercolor*{block body example}{bg=structure.bg!90!black}
\setbeamercolor*{block title}{parent=structure,bg=structure.bg!75!black}
\setbeamercolor*{block title alerted}{use={structure,alerted text},fg=alerted text.fg!75!structure.fg,bg=structure.bg!75!black}
\setbeamertemplate{navigation symbols}{}
\setbeamertemplate{items}[square]
\setbeamercolor{item projected}{fg=white}
\setbeamercolor*{normal text}{fg=white!90!blue,bg=blue!70!black}
\setbeamercolor*{separation line}{}
\setbeamercolor*{fine separation line}{}
\setbeamercolor{alerted text}{fg=yellow}

%\beamerdefaultoverlayspecification{<+->}




\usepackage[italian]{babel}
\usepackage[utf8]{inputenc}
\usepackage{pgf}
\usepackage{verbatim}

% \usepackage{euler}
\usepackage[T1]{fontenc}

\author{Gianluca Della Vedova}
\title{Laboratorio di Informatica}
\institute{Univ. Milano--Bicocca\\
  \texttt{https://gianluca.dellavedova.org}}
\date{\today}


% If you wish to uncover everything in a step-wise fashion, uncomment
% the following command:
%\beamerdefaultoverlayspecification{<+->}


\begin{document}

\begin{frame}
  \titlepage
\end{frame}


\begin{frame}\frametitle{Gianluca Della Vedova}
  \begin{itemize}
  \item
    Ufficio U14-2041
  \item
    \textsf{\small https://gianluca.dellavedova.org}
  \item
    \textsf{\small gianluca.dellavedova@unimib.it}
  \end{itemize}
\end{frame}

% \begin{frame}
%   \frametitle{Consensus Clustering}
%   \begin{block}{Problem}
%     \alert{Input}: set $\Pi$ of partitions of $U$\\
%     \alert{Output}: a partition $P$ of $U$\\
%     \alert{Goal}: $P$ the best possible representative of $\Pi$
%   \end{block}

%   \begin{block}{Objective Function}
%     \begin{itemize}
%     \item
%       Symmetric difference between two partitions =
%       Pairs of elements clustered differently
%     \item
%       $d(P,\Pi)=\sum_{\pi\in \Pi}d(\pi,P)$
%     \end{itemize}
%   \end{block}
% \end{frame}



\begin{frame}\frametitle{Finestre}
  \begin{itemize}
  \item
    Editor
  \item
    Log
  \item
    Output
  \item
    Icona Esegui
  \end{itemize}
\end{frame}

\begin{frame}\frametitle{Data set}
  \begin{itemize}
  \item
    righe = osservazioni
  \item
    colonne = variabili
  \end{itemize}
\end{frame}

\begin{frame}[fragile]\frametitle{Data Step}
  \VerbatimInput{code/p1.sas}
\end{frame}

\begin{frame}\frametitle{Data Step: dati nel programma}
  \VerbatimInput{code/p0.sas}
\end{frame}


\begin{frame}\frametitle{Valori mancanti}
  \begin{itemize}
  \item
    Rappresentati da un punto \texttt{.}
  \item
    Inseriti come spazi o punti
  \end{itemize}\end{frame}

\begin{frame}\frametitle{Data Step: Importazione}
  \begin{itemize}
  \item
    Da file di dati grezzi (.txt .csv .dat)
  \item
    Da wizard di importazione
  \end{itemize}
\end{frame}

\begin{frame}\frametitle{Data set}
  \begin{itemize}
  \item
    nome di data set $\le$ 32 caratteri
  \item
    lettere e cifre
  \item
    cifre in fondo
  \item
    nome di variabili $\le$ 32 caratteri
  \end{itemize}
\end{frame}


\begin{frame}\frametitle{Filesystem}
  \begin{itemize}
  \item
    Ogni data set corrisponde ad un file
  \item
    Una directory (cartella) contenente data set = \alert{libreria}
  \item
    \texttt{LIBNAME nomelibreria directory;}
  \item
    data set permamente = in una libreria
  \item
    data set temporaneo = libreria \texttt{WORK} = no libreria
  \item
    nome libreria $\le$ 8 caratteri
  \end{itemize}\end{frame}

\begin{frame}\frametitle{Filesystem}
  \begin{itemize}
  \item
    Ogni data set corrisponde ad un file
  \item
    Dati: data set oppure file di dati grezzi
  \end{itemize}
\end{frame}

\begin{frame}[fragile]\frametitle{Data Step: da dati grezzi}
  \VerbatimInput{code/p2.sas}
\end{frame}

\begin{frame}[fragile]\frametitle{Data Step: da dati grezzi 2}
  \VerbatimInput{code/p3.sas}
  Dati separati da spazi
\end{frame}


\begin{frame}[fragile]\frametitle{Data Step - lettura da file }
  \VerbatimInput{code/p5.sas}
  Dati separati da virgola

  \texttt{DSD}: campi alfanumerici racchiusi da virgolette
\end{frame}

\begin{frame}[fragile]\frametitle{Data Step - lettura da file }
  \VerbatimInput{code/p6.sas}
  Dati separati da tabulazione
\end{frame}

\begin{frame}\frametitle{Data Step - lettura da file }
  \begin{itemize}
  \item
    \texttt{FIRSTOBS}: da quale riga iniziare
  \item
    \texttt{OBS}: quante righe leggere
  \item
    \texttt{MISSOVER}: variabili non assegnate=MISSING
  \item
    \texttt{TRUNCOVER}: dati a fondo riga = scartati
  \end{itemize}
\end{frame}

\begin{frame}[fragile]\frametitle{Formato a colonne}
  \VerbatimInput{code/p7.sas}

  \begin{itemize}
  \item
    Il cognome inizia alla colonna n. 11 e termina alla colonna n. 20.
  \item
    Righello nel log
  \item
    Gli spazi non separano.
  \end{itemize}
\end{frame}


\begin{frame}[fragile]\frametitle{Formati Alfanumerici}
  \verb!: $20.!

  Rappresenta il numero massimo di caratteri di un campo (in questo caso
  20). Il dato termina con il primo spazio.

  \verb!$20.!

  Esattamente 20 caratteri.
\end{frame}

\begin{frame}[fragile]\frametitle{Formati Numerici}
  \verb!10.3!

  Numero totale di caratteri/cifre

  .

  Numero di cifre decimali
\end{frame}



\begin{frame}[fragile]\frametitle{Formati Date}
  \begin{itemize}
  \item
    \verb!DDMMYY8.!
  \item
    \verb!DATE7.!
  \item
    Le date sono rappresentate in un formato interno, come numero di giorni dall'1/1/1960.
  \item
    Quindi una data {\`e} un numero
  \item
    Differenza fra date
  \end{itemize}
\end{frame}

\begin{frame}[fragile]\frametitle{Funzioni su numeri}
  \begin{itemize}
  \item
    \verb!SUM(var1 var2 var3)!,  \verb!MIN(var1 var2 var3)!,
    \verb!MAX(var1 var2 var3)!,     \verb!MEAN(var1 var2 var3)!,
  \item
    \verb!SUM(of var1-var3)!
  \item
    funzioni su variabili numeriche.
  \item
    Attenzione ai valori mancanti.
  \end{itemize}
\end{frame}

\begin{frame}[fragile]\frametitle{Funzioni su numeri}
  \begin{itemize}
  \item
    concatena due stringhe:     \verb!CAT(var1, var2)! 
  \item
    estrae sottostringa:
    \verb!SUBSTR(var, inizio, lunghezza)!
  \item
    converte in formato numerico:
    \verb!INPUT(var, informat)!
  \item
  trova una sottostringa:
    \verb!INDEX(var, sottostringa)!
  \item
  calcola la lunghezza:
    \verb!LENGTH(var)!
  \item
  trasforma in maiuscolo:
    \verb!UPCASE(var)!
  \end{itemize}
\end{frame}

\begin{frame}[fragile]\frametitle{Altre funzioni}
  \begin{itemize}
  \item
    \verb!MONTH(data)!
  \item
    \verb!YEAR(data)!,  \verb!DAY(data)!
  \item
    Sono funzioni che agiscono sulle date
  \item
    \verb!'01Mar2000'd!
  \end{itemize}
\end{frame}



\begin{frame}[containsverbatim]\frametitle{Formati}
  \VerbatimInput{code/p8.sas}

  \begin{itemize}
  \item
    le parentesi raggruppano variabili
  \item
    \verb!COMMA7.! legge numeri con virgole come separatore migliaia
  \item
    \verb!DOLLAR7.2! come \verb!COMMA!, ma con un dollaro all'inizio
  \item
    \verb!@30! va a colonna 30
  \end{itemize}
\end{frame}



\begin{frame}[containsverbatim]\frametitle{Riepilogo data set}
  \VerbatimInput{code/p19.sas}
\end{frame}


\begin{frame}[fragile]\frametitle{Data Step - copia }
  \VerbatimInput{code/p9.sas}
\end{frame}


\begin{frame}[fragile]\frametitle{Data Step - copia }
  \VerbatimInput{code/p10.sas}
\end{frame}



\begin{frame}[fragile]\frametitle{Creazione variabili}
  \VerbatimInput{code/p11.sas}

  \begin{itemize}
  \item
    Le operazioni fra variabili creano nuove variabili
  \item
    Operazioni usuali, \verb!+! \verb!-! \verb!/! \verb!*! \verb!**!
  \end{itemize}
\end{frame}




\begin{frame}[containsverbatim]\frametitle{Selezione di osservazioni}
  \begin{itemize}
  \item
    \verb!IF condizione;!
  \item
    \VerbatimInput{code/p12.sas}
  \item
    Operatori confronto, \texttt{EQ $=$, GT $>$, GE $\geq$, LT $<$, LE $\leq$, NE $\neq$}
  \item
    Valori non numerici devono essere racchiusi da apici
  \end{itemize}\end{frame}

\begin{frame}[containsverbatim]\frametitle{Selezione di osservazioni 2}
  \VerbatimInput{code/p12a.sas}
  \begin{itemize}
  \item
    Rimozione di osservazione
  \end{itemize}\end{frame}

\begin{frame}[containsverbatim]\frametitle{Istruzioni condizionali}
  \VerbatimInput{code/p13.sas}
  \begin{itemize}
  \item
    Pi{\`u} semplice se una sola istruzione
  \item
    \verb!IF condizione THEN istruzione;!
  \end{itemize}
\end{frame}

\begin{frame}[containsverbatim]\frametitle{Condizioni complesse}
  \VerbatimInput{code/p14.sas}
  \begin{itemize}
  \item
    Operatori logici: AND, OR, NOT
  \end{itemize}
\end{frame}

\begin{frame}[containsverbatim]\frametitle{Condizioni complesse 2}
  \VerbatimInput{code/p14a.sas}
  \begin{itemize}
  \item
    Cosa succede se un'osservazione non soddisfa alcuna condizione?
  \item
    E se ne soddisfa pi{\`u} di una?
  \end{itemize}
\end{frame}






\begin{frame}[containsverbatim]\frametitle{Esportare un dataset}
Solo per ottenere un file grezzo a partire da un dataset.
Sono istruzioni simmetriche a \verb|INFILE| e \verb+PUT+, quindi devono essere
all'interno di un data step.

\begin{itemize}
\item
\verb+FILE+: indica il file che verrà ottenuto e contiene l'opzione \verb+DLM+.
\item
\verb+PUT+: specifica le variabili da esportare, il loro ordine e il formato
    delle variabili.
\end{itemize}
\end{frame}



\begin{frame}[fragile]\frametitle{Data Step - copia }
  \VerbatimInput{code/p10a.sas}
  \begin{itemize}
  \item
    Parte di data step eseguita per ogni osservazione del data set di origine
  \item
    Ciclo implicito
  \item variabili inizialmente mancanti
  \end{itemize}
\end{frame}

% \begin{frame}[fragile]\frametitle{Retain}
%   \begin{itemize}
%   \item
%     \verb+RETAIN+ mantiene il valore di una variabile per osservazioni diverse
%   \item
%     senza \verb+RETAIN+: variabile MISSING
%   \item
%     con \verb+RETAIN+: mantiene valore precedente
%   \item
%     Somma $\Rightarrow$ \verb+RETAIN+
%   \item
%     \verb!a + (5 * b)!
%   \item
%     \verb!a+1;!
%   \end{itemize}
% \end{frame}




% \begin{frame}[containsverbatim]\frametitle{Sequenza di variabili}
%   \begin{itemize}
%   \item
%     \verb+temp1 temp2 temp3 temp4 temp5+
%   \item
%     \verb+temp1-temp5+
%   \item
%     Sono variabili individuali
%   \end{itemize}\end{frame}


% \begin{frame}[containsverbatim]\frametitle{Array}
%   \begin{itemize}
%   \item
%     Elemento array = nome alternativo variabile
%   \item
%     \verb!temp[1]!
%   \item
%     nome collettivo dell'array + indice
%   \item
%     \verb!array temp[5];!
%   \item
%     Gli array devono essere \alert{dichiarati} con il numero di elementi.
%   \end{itemize}\end{frame}


% \begin{frame}[containsverbatim]\frametitle{Array}
%   \begin{itemize}
%   \item
%     \texttt{temp[1]} singolo elemento dell'array
%   \item
%     \texttt{temp} nome array
%   \item
%     \verb!array temp[5];!
%   \item
%     Associa le variabili \verb!temp1-temp5! agli elementi dell'array \verb!temp!.
%   \item
%     \verb!array temp[5] a b c d e;!
%   \item
%     Associa le variabili \verb!a b c d e! agli elementi dell'array \verb!temp!.
%   \end{itemize}\end{frame}


% \begin{frame}[containsverbatim]\frametitle{Ciclo}
%   \begin{itemize}
%   \item
%     Ripetere pi{\`u} volte delle istruzioni.
%   \item
%     numero di volte = contatore = variabile dedicata
%   \item
%     Ciclo \verb+DO+
%   \end{itemize}
%   \VerbatimInput{code/p18.sas}
% \end{frame}



% \begin{frame}[containsverbatim]\frametitle{Gestione variabili}
%   \begin{itemize}
%   \item
%     Il corpo di un data set viene ripetuto per ogni osservazione del data set originario.
%   \item
%     Il contenuto delle variabili vengono distrutte all'inizio dell'esecuzione di ogni osservazione.
%   \item
%     \verb!i+1;!
%   \item
%     \verb!RETAIN variabile;!
%   \end{itemize}\end{frame}


\begin{frame}[containsverbatim]\frametitle{Stampa}
  \VerbatimInput{code/p15.sas}
  \begin{itemize}
  \item
    Stampa il contenuto dell'ultimo data set creato.
  \end{itemize}\end{frame}

\begin{frame}[containsverbatim]\frametitle{Stampa}
  \VerbatimInput{code/p16.sas}
  \begin{itemize}
  \item
    Stampa solo le variabili specificate
  \end{itemize}\end{frame}

\begin{frame}[containsverbatim]\frametitle{Stampa}
  \VerbatimInput{code/p17.sas}
  \begin{itemize}
  \item
    Usa la variabile \texttt{cognome} al posto di \texttt{obs}.
  \item
    Definisce il titolo.
  \end{itemize}\end{frame}

\begin{frame}[containsverbatim]\frametitle{Stampa}
  \VerbatimInput{code/p17a.sas}
  \begin{itemize}
  \item
    \verb+WHERE+ definisce su quali osservazioni agire
  \end{itemize}\end{frame}

\begin{frame}[containsverbatim]\frametitle{Etichette di stampa}
  \VerbatimInput{code/p34.sas}

  \vspace{1em}
  L'etichetta della variabile deve essere dentro il \verb!DATA! step
\end{frame}




\begin{frame}[containsverbatim]\frametitle{Formati di stampa}
  \begin{itemize}
  \item
    \verb!FORMAT variabile 16.;!
  \item
    \verb!FORMAT variabile DATE7.;!
  \item
    \verb!FORMAT variabile 8.2;!
  \item
    La \verb!FORMAT! deve essere dentro la \verb!PROC PRINT!
  \end{itemize}
\end{frame}

\begin{frame}[containsverbatim]\frametitle{Formati di stampa}
  \VerbatimInput{code/p17b.sas}
\end{frame}



\begin{frame}[containsverbatim]\frametitle{Ordinare un data set}
  \VerbatimInput{code/p20.sas}
\end{frame}

% \begin{frame}\frametitle{Ordinare = Partizionare}
%   \begin{itemize}
%   \item
%     Stesso valore = osservazioni consecutive = insieme nella partizione
%   \item
%     \texttt{FIRST.variabile} =
%     prima osservazione dell'insieme
%   \item
%     \texttt{LAST.variabile} =
%     ultima osservazione dell'insieme
%   \end{itemize}\end{frame}

% \begin{frame}\frametitle{Uso di first  e last}
%   \begin{itemize}
%   \item
%     Sono predicati
%   \item
%     Vengono utilizzati in \texttt{IF}
%   \end{itemize}\end{frame}

% \begin{frame}[containsverbatim]\frametitle{Esempio}
%   \VerbatimInput{code/p21.sas}
% \end{frame}



% \begin{frame}\frametitle{Raggruppare osservazioni 1}
%   \begin{itemize}
%   \item
%     Una variabile \alert{chiave}
%   \item
%     Ordinare il data set con \texttt{BY chiave}
%   \item
%     Gestire un contatore \texttt{i} per contare la posizione dell'osservazione
%     corrente all'interno del gruppo.
%   \end{itemize}\end{frame}


% \begin{frame}\frametitle{Raggruppare osservazioni 2}
%   \begin{itemize}
%   \item
%     \texttt{FIRST.chiave}: azzerare  \texttt{i}
%   \item
%     Ogni osservazione: scrivere il dato letto nella i-esima posizione di un array,
%     incrementare  \texttt{i}
%   \item
%     \texttt{LAST.chiave}: \texttt{OUTPUT}
%   \end{itemize}\end{frame}


% \begin{frame}[containsverbatim]\frametitle{Raggruppare osservazioni 3}
%   \small
%   Provincia\hspace{2ex}temp\hspace{1ex} i\hspace{1ex} FIRST.provincia\hspace{1ex} LAST.provincia
% \begin{verbatim}
% Milano    23
% Milano,   20
% Milano,   22
% Milano,   12
% Milano,   .
% Pavia,    24
% Pavia,    21
% Pavia,    19
% Pavia,    24
% Pavia,    21
% \end{verbatim}
% \end{frame}

% \begin{frame}[containsverbatim]\frametitle{Raggruppare osservazioni 3}
%   \small
%   Provincia\hspace{2ex}temp\hspace{1ex} i\hspace{1ex} FIRST.provincia\hspace{1ex} LAST.provincia
% \begin{verbatim}
% Milano    23  1       V
% Milano    20  2
% Milano    22  3
% Milano    12  4
% Milano    .   5                   V
% Pavia     24  1       V
% Pavia     21  2
% Pavia     19  3
% Pavia     24  4
% Pavia     21  5                   V
% \end{verbatim}
% \end{frame}


\begin{frame}[containsverbatim]\frametitle{Nuovi formati}
  \VerbatimInput{code/p35.sas}
  \begin{itemize}
  \item
    Adesso \texttt{formato.} {\`e} utilizzabile in una istruzione  \texttt{format}.
  \end{itemize}
\end{frame}


\begin{frame}[containsverbatim]\frametitle{Nuovi formati}
  \VerbatimInput{code/p36.sas}
\end{frame}


\begin{frame}[containsverbatim]\frametitle{Nuovi formati}
  \VerbatimInput{code/p37.sas}
\end{frame}




\begin{frame}[containsverbatim]\frametitle{Calcolare statistiche}
  \VerbatimInput{code/p22.sas}
  \begin{itemize}
  \item
    Seleziona variabili
  \item
    Altrimenti su tutte le variabili numeriche
  \end{itemize}
\end{frame}



\begin{frame}[containsverbatim]\frametitle{Calcolare statistiche}
  \VerbatimInput{code/p23.sas}
  \begin{itemize}
  \item
    Stratifica le statistiche
  \end{itemize}
\end{frame}


\begin{frame}[containsverbatim]\frametitle{Calcolare statistiche}
  \VerbatimInput{code/p24.sas}
  \begin{itemize}
  \item
    Risultati in un dataset
  \item
    \verb+_TYPE_+ e \verb+_FREQ_+
  \end{itemize}
\end{frame}

\begin{frame}[containsverbatim]\frametitle{Calcolare statistiche}
  \VerbatimInput{code/p24a.sas}
  \begin{itemize}
  \item
    \verb+NWAY+: Stratificazione massima
  \end{itemize}
\end{frame}

\begin{frame}\frametitle{Statistiche}
  \texttt{N}: numero osservazioni con valore non mancante\\
  \texttt{NMISS}: numero osservazioni con valore mancanti\\
  \texttt{NONOBS}: numero osservazioni\\
  \texttt{MEAN}: media\\
  \texttt{MEDIAN}: mediana\\
  \texttt{STDDEV}: deviazione standard\\
  \texttt{MAX}: massimo\\
  \texttt{SUM}: somma\\
  \texttt{ALPHA=.05 CLM}: intervallo di confidenza\\
\end{frame}


\begin{frame}[containsverbatim]\frametitle{Fondere due data set}
  \VerbatimInput{code/p28a.sas}
  \begin{itemize}
  \item
    Le osservazioni vengono aggiunte sequenzialmente
  \end{itemize}
\end{frame}



\begin{frame}[containsverbatim]\frametitle{Fondere due data set}
  \VerbatimInput{code/p28.sas}
  \begin{itemize}
  \item
    Le osservazioni sono mantenute
  \item
    Il campo comune in entrambi i data set guida la fusione.
  \item
    \verb+BY+ richiede un ordinamento.
  \end{itemize}
\end{frame}






% \begin{frame}[containsverbatim]\frametitle{Grafici}
%   \VerbatimInput{code/p.sas}
%   PLOT variabile1*variabile2;
%   RUN;
% \end{verbatim}
%        Grafico in output con le due variabili
%      \end{frame}


%      \begin{frame}[containsverbatim]\frametitle{Grafici}
%        \VerbatimInput{code/p.sas}
%        PLOT variabile1*variabile2
%        variabile3*variabile2;
%        /*
%        PLOT (variabile1 variabile3)
%        *variabile2;
%        */
%        RUN;
% \end{verbatim}
%        Due grafici in output
%      \end{frame}


%      \begin{frame}[containsverbatim]\frametitle{Overlay}
%        \VerbatimInput{code/p.sas}
%        PLOT (variabile1 variabile3)
%        *variabile2 /OVERLAY;
%        RUN;
% \end{verbatim}
%        Due grafici sovrapposti in uno solo
%      \end{frame}

%      \begin{frame}[containsverbatim]\frametitle{Grafici a bolle}
%        \VerbatimInput{code/p.sas}
%        BUBBLE variabile1*variabile2=tipo;
%        RUN;
% \end{verbatim}
%        Grafico in output con le due variabili, \texttt{tipo} viene
%        rappresentata con un cerchio.
%      \end{frame}

%      \begin{frame}[containsverbatim]\frametitle{Grafici a barre}
%        \VerbatimInput{code/p.sas}
%        BLOCK variabile1 variabile2
%        /DISCRETE;
%        RUN;
% \end{verbatim}
%        Grafico in output con le due variabili, ottimizzato per valori discreti.
%      \end{frame}

%      \begin{frame}\frametitle{Opzioni}
%        \begin{itemize}
%        \item
%          \texttt{TITLE1} o \texttt{TITLE2}: titolo
%        \item
%          \texttt{SYMBOL1}: per definire il simbolo del primo grafico
%        \item
%          \texttt{AXIS1}: per definire gli assi cartesiani
%        \end{itemize}\end{frame}

%      \begin{frame}\frametitle{Opzioni}
%        \texttt{goptions reset=global gunit=pct border cback=white\\
%        colors=(black blue green red)\\
%        ftitle=swissb ftext=swiss htitle=6 htext=4;
%      }

%        \vspace{1ex}
%        \texttt{symbol1 interpol=join value=diamond height=3 color=red;}
%      \end{frame}



% %      \begin{frame}[containsverbatim]\frametitle{Esempio}
%        \VerbatimInput{code/p.sas}
%        title 'Andamento esami Lab. Stat.-Inf.';
%        symbol1 color=red interpol=join value=dot height=2;
%        axis1 order=(2002 to 2005 by 1);
% \end{verbatim}
%      \end{frame}


% \begin{frame}[containsverbatim]\frametitle{Correlazione lineare}
%   \VerbatimInput{code/p25.sas}
%   \begin{itemize}
%   \item
%     Nessuna distinzione fra variabili dipendenti e indipendenti
%   \end{itemize}
% \end{frame}


% \begin{frame}[containsverbatim]\frametitle{Correlazione lineare}
%   \VerbatimInput{code/p26.sas}
%   \begin{itemize}
%   \item
%     \verb!WITH!: variabili dipendenti
%   \item
%     \verb!VAR!: variabili indipendenti
%   \end{itemize}
% \end{frame}


% \begin{frame}[containsverbatim]\frametitle{Correlazione lineare}
%   \VerbatimInput{code/p27.sas}
%   \begin{itemize}
%   \item
%     \verb!BEST!
%   \end{itemize}
% \end{frame}





% \begin{frame}[containsverbatim]\frametitle{Regressione lineare}
%   \VerbatimInput{code/p32.sas}
%   \begin{itemize}
%   \item
%     Calcola la migliore retta della forma $vary=a*varx+b$ per approssimare
%     i dati.
%   \end{itemize}
% \end{frame}

% \begin{frame}[containsverbatim]\frametitle{Grafico di regressione}
%   \VerbatimInput{code/p33.sas}
%   \begin{itemize}
%   \item
%     Disegna la migliore retta della forma $vary=a*varx+b$ per approssimare
%     i dati.
%   \end{itemize}
% \end{frame}

% \begin{frame}[containsverbatim]\frametitle{Grafico di regressione}
%   \VerbatimInput{code/p33a.sas}
%   \begin{itemize}
%   \item
%     Disegna sia i punti del data set che la retta di regressione
%   \end{itemize}
% \end{frame}




\begin{frame}[containsverbatim]\frametitle{Analisi delle frequenze}
  \VerbatimInput{code/p29.sas}
\end{frame}


\begin{frame}[containsverbatim]\frametitle{Analisi delle frequenze}
  \VerbatimInput{code/p30.sas}
\end{frame}

\begin{frame}[containsverbatim]\frametitle{Analisi delle frequenze}
  \VerbatimInput{code/p31.sas}
  \begin{itemize}
  \item
    \verb+WEIGHT+: variabile che indica un peso per ogni osservazione
  \item
    \verb+WEIGHT+: variabile quantitativa
  \item
    \verb+TABLES+: variabili qualitative
  \end{itemize}
\end{frame}




\begin{frame}[containsverbatim]\frametitle{Output Delivery System: ODS}
  \begin{itemize}
  \item
    Le procedure inviano dati all'ODS
  \item
    Destinazioni: \verb+LISTING+ (standard output), \verb+HTML+, \verb+PDF+,
    \verb+OUTPUT+ (data set), \verb+MARKUP+ (csv, xml,\ldots), \verb+DOCUMENT+
  \item
    Template: tabella, stile
  \end{itemize}
\end{frame}


\begin{frame}[containsverbatim]\frametitle{Output Delivery System}
  \VerbatimInput{code/p39.sas}
  \begin{itemize}
  \item
    Per avere nel log come ODS interpreta il programma
  \end{itemize}
\end{frame}


\begin{frame}[containsverbatim]\frametitle{Output Delivery System}
  \VerbatimInput{code/p38a.sas}
  \begin{itemize}
  \item
    Seleziona elemento
  \item
    Evitare nome duplicati (Path)
  \item
    \verb+ODS EXCLUDE+
  \end{itemize}
\end{frame}

\begin{frame}[containsverbatim]\frametitle{Output Delivery System}
  \VerbatimInput{code/p38.sas}
\end{frame}


\begin{frame}[containsverbatim]\frametitle{Output Delivery System}
  \VerbatimInput{code/p40.sas}
\end{frame}


\begin{frame}[containsverbatim]\frametitle{Output Delivery System}
  \VerbatimInput{code/p40a.sas}
\end{frame}

\begin{comment}
  \begin{frame}[containsverbatim]\frametitle{Grafici}
    \VerbatimInput{code/grafici1.sas}
  \end{frame}

  \begin{frame}[containsverbatim]\frametitle{Grafici 2}
    \VerbatimInput{code/grafici2.sas}
  \end{frame}

  \begin{frame}[containsverbatim]\frametitle{Grafici e Proc Corr}
    \VerbatimInput{code/grafici3.sas}
  \end{frame}


  \begin{frame}[containsverbatim]\frametitle{Grafici e Proc Reg}
    \VerbatimInput{code/grafici4.sas}
  \end{frame}
\end{comment}





% \begin{frame}\frametitle{Macro}
%   \begin{itemize}
%   \item
%     Macro = sostituzione con altro testo
%   \item
%     Macro = programma che scrive programmi
%   \item
%     Sostituzione delle macro: prima dell'esecuzione del programma
%   \end{itemize}
% \end{frame}

% \begin{frame}[containsverbatim]\frametitle{Macro variabile}
%   \VerbatimInput{code/p41.sas}
%   \begin{itemize}
%   \item
%     Usato in
%   \end{itemize}
%   \VerbatimInput{code/p42.sas}
% \end{frame}


% \begin{frame}[containsverbatim]\frametitle{Macro programmi}
%   \begin{itemize}
%   \item
%     Programmi che scrivono programmi
%   \end{itemize}
%   \VerbatimInput{code/p43.sas}
%   \begin{itemize}
%   \item
%     La macro viene invocata con
%   \end{itemize}
%   \VerbatimInput{code/p44.sas}
% \end{frame}

% \begin{frame}[containsverbatim]\frametitle{Esempio Macro programma}
%   \VerbatimInput{code/p45.sas}
%   \begin{itemize}
%   \item
%     Esempio di utilizzo
%   \end{itemize}
%   \VerbatimInput{code/p45a.sas}
% \end{frame}





% \begin{frame}[containsverbatim]\frametitle{Macro vs. programmi}
%   \begin{itemize}
%   \item
%     \verb+%IF+..\verb+%THEN+..\verb+%ELSE+.. vs \verb+IF+..\verb+THEN+..\verb+ELSE+..
%   \item
%     \verb+%DO+ vs \verb+DO+
%   \item
%     In una macro si pu{\`o} includere un intero data step o proc step.
%   \end{itemize}
% \end{frame}

% \begin{frame}[containsverbatim]\frametitle{Esempio Macro programma (2)}
%   \VerbatimInput{code/p46.sas}
% \end{frame}

% \begin{frame}[containsverbatim]\frametitle{Esempio Macro programma (2)}
%   \VerbatimInput{code/p46a.sas}
% \end{frame}


% \begin{frame}[containsverbatim]\frametitle{Esempio}
%   \begin{itemize}
%   \item
%     Ordinare un data set solo se un parametro vale 1
%   \end{itemize}
%   \VerbatimInput{code/p48.sas}
%   \begin{itemize}
%   \item
%     Invocato con
%   \end{itemize}
%   \VerbatimInput{code/p49.sas}
% \end{frame}



% \begin{frame}[containsverbatim]\frametitle{CALL SYMPUT}
%   \begin{itemize}
%   \item
%     Leggere i dati da un data set
%   \item
%     Non usare nello stesso DATA STEP di definizione
%   \end{itemize}
%   \VerbatimInput{code/p50.sas}
% \end{frame}


\begin{frame}[containsverbatim]\frametitle{Licenza d'uso}
  \small
Quest'opera è distribuita con Licenza Creative Commons
Attribuzione - Condividi allo stesso modo 4.0 Internazionale
\url{http://creativecommons.org/licenses/by-sa/4.0/}.

La versione più recente, con i sorgenti per modificare l'opera si trova
a \url{https://github.com/gdv/laboratorio-informatica}.

\end{frame}

\end{document}




%%% Local Variables:
%%% mode: latex
%%% TeX-PDF-mode: t
%%% buffer-file-coding-system: utf-8
%%% End:
